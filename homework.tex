%!TEX TS-program = xelatex
%!TEX encoding = UTF-8 Unicode

\documentclass{homework}

%----------------------------------------------------------%
%------------------NAME AND CLASS SECTION------------------%
%----------------------------------------------------------%

\newcommand{\hmwkTitle}{Assignment\ \#1}
\newcommand{\hmwkDueDate}{Monday,\ July\ 18,\ 2013}
\newcommand{\hmwkClass}{Class Here}
\newcommand{\hmwkClassTime}{1:15pm}
\newcommand{\hmwkInstructor}{Instructor}
\newcommand{\hmwkAuthorName}{Name Here} 

\begin{document}
\maketitle

%----------------------------------------------------------%
%---------TABLE OF CONTENTS - Uncomment to include---------%
%----------------------------------------------------------%

%\setcounter{tocdepth}{1} % Uncomment to remove subsections
%\newpage
%\tableofcontents
%\newpage

%----------------------------------------------------------%
%-------------------PROBLEM 1 -----------------------------%
%----------------------------------------------------------%

\begin{chapter}{Section 12.1}

\begin{problem}%[Problem \Roman{homeworkProblemCounter} (14.)] 
%Uncomment to use Roman numerals for problems. Substitute in \arabic for \Roman for regular numerals with (14.) after it.

\lipsum[1] % Used to create loren ipsum dummy text

$$y=3x^2+4x+4$$

\vspace{-5pt} \begin{probpart}{(a)}
 \lipsum[2]

\answer{ 
% Answer 
} \end{probpart}

\begin{probpart}{(b)}
\lipsum[3]

\answer{ 
% Answer 
} \end{probpart}

\begin{probpart}{(c)}
\lipsum[9]

\answer{ 
% Answer
} \end{probpart}

\begin{probpart}{(d)}
The estimated results using 1000 observations are given in Table 5.9.

\answer{ 
\lipsum[10]
} \end{probpart}

\end{problem}

%----------------------------------------------------------%
%-------------------PROBLEM 2 -----------------------------%
%----------------------------------------------------------%

\begin{problem}

\lipsum[5]

\vspace{5pt} 
\begin{probpart}{(i)}

\lipsum[6]

\answer{ 
% Answer
} \end{probpart}

\begin{probpart}{(ii)}
\lipsum[8]

\answer{ 
% Answer
} \end{probpart}

\begin{probpart}{(iii)}
\lipsum[7]

\answer{
\lipsum[10-11]
} 

\answer{
\lipsum[10-12]
} \end{probpart}

\end{problem} 

\end{chapter}

%----------------------------------------------------------%
%------------------ PROBLEM 3 -----------------------------%
%----------------------------------------------------------%

\begin{chapter}{Section 12.2}

\begin{problem} 

\lipsum[13]

\answer{
\lipsum[14-15]
} 

\end{problem}

\end{chapter}

\end{document}
